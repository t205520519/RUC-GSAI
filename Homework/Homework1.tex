\documentclass[12pt]{article}
\usepackage{fancyhdr}  % 用于页眉页脚定制
\usepackage{graphicx}  % 用于插入图片
\usepackage{lipsum}    % 用于生成示例文本
\usepackage[UTF8]{ctex}
\usepackage{geometry}
\geometry{a4paper,margin=1in}
\usepackage{array} 
\usepackage{listings}
\usepackage{xcolor}
\usepackage{enumitem}

% 配置颜色
\definecolor{codegreen}{rgb}{0,0.6,0}
\definecolor{codegray}{rgb}{0.5,0.5,0.5}
\definecolor{codepurple}{rgb}{0.58,0,0.82}
\definecolor{backcolour}{rgb}{0.95,0.95,0.92}

% 配置代码高亮风格
\lstdefinestyle{mystyle}{
    backgroundcolor=\color{backcolour}, 
    commentstyle=\color{codegreen},
    keywordstyle=\color{magenta},
    numberstyle=\tiny\color{codegray},
    stringstyle=\color{codepurple},
    basicstyle=\ttfamily\footnotesize,
    breakatwhitespace=false,         
    breaklines=true,                 
    captionpos=b,                    
    keepspaces=true,                 
    numbers=left,                    
    numbersep=5pt,                  
    showspaces=false,                
    showstringspaces=false,
    showtabs=false,                  
    tabsize=2
}

% 自定义Python语言的关键字
\lstset{language=Python,
    style=mystyle,
    escapeinside={(*@}{@*)}, 
    morekeywords={*,dictionaries,*},
    emph={MyClass,self}, emphstyle=\color{codepurple}\ttfamily,
}




% 定制页脚
\pagestyle{fancy}
\fancyhf{}  % 清除默认页眉页脚内容
\fancyfoot[C]{\raisebox{-0.5cm}{\includegraphics[width=4cm]{GSAI.png}}}  % 在页脚中居中放置图片
\fancyfoot[R]{\thepage}  % 在页脚右侧显示页码
%报告信息

\begin{document}

\begin{titlepage}
    \centering
    \vspace*{3cm}  % 在顶部留出3厘米的空间
    {\Huge\bf Homework1}  % 标题,字号为Huge
    \vspace{1cm}\\  % 标题与姓名之间的垂直间距
    \vspace{3cm}
    {\Large 姓名: 田原}  % 大字号显示姓名
    \vspace{1cm}  % 姓名与学号之间的垂直间距
    
    {\Large 学号: 2023200406}
    \vspace{1cm}  % 学号与专业之间的垂直间距
    
    
    {\Large \today}  % 显示当前日期
    
    \vfill  % 将剩余空间分配到页面底部
\end{titlepage}
\newpage

\newpage

\section{题目1}



\begin{table}[htbp]
\centering
\begin{tabular}{|>{\centering\arraybackslash}m{4cm}|>{\centering\arraybackslash}m{4cm}|>{\centering\arraybackslash}m{4cm}|>{\centering\arraybackslash}m{4cm}|}
\hline
\textbf{Binary} & \textbf{Octal} & \textbf{Decimal} & \textbf{Hexadecimal} \\
\hline
10101010110 & 2526 & 1318 & 0x556 \\
\hline
111111111& 777 & 511 & 0x1ff \\
\hline
111000101 & 705 & 453 & 0x1c5 \\
\hline
11111011111& 3737 & 2015 & 0x7df\\
\hline
10000001101 & 10032 & 1037 & 0x40d\\
\hline
\end{tabular}
\caption{一个6行4列的表格示例}
\label{tab: 6x4table}
\end{table}

\section{题目2}
\begin{enumerate}[label*=(\alph*)]
    \item 0x3a
    \item 0xff
    \item 0xc5
    \item 0xc5
    \item true
    \item true
\end{enumerate}
\section{题目3}

\begin{enumerate}[label*=(\alph*)]
    \item !x\,\^\,0
    \item !x
    \item !x\,\^\,0xFF
    \item !x\&0xFF
    
\end{enumerate}

\section{题目4}
(x\&0xFFFF)|(y\&\,\~\,0xFFFF)


\end{document}


